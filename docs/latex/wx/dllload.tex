%%%%%%%%%%%%%%%%%%%%%%%%%%%%%%%%%%%%%%%%%%%%%%%%%%%%%%%%%%%%%%%%%%%%%%%%%%%%%%%
%% Name:        dllload.tex
%% Purpose:     wxDllLoader documentation
%% Author:      Vadim Zeitlin
%% Modified by:
%% Created:     02.04.00
%% RCS-ID:      $Id$
%% Copyright:   (c) Vadim Zeitlin
%% License:     wxWindows license
%%%%%%%%%%%%%%%%%%%%%%%%%%%%%%%%%%%%%%%%%%%%%%%%%%%%%%%%%%%%%%%%%%%%%%%%%%%%%%%

\section{\class{wxDllLoader}}\label{wxdllloader}

wxDllLoader is a class providing an interface similar to Unix's {\tt
dlopen()}. It is used by the wxLibrary framework and manages the actual
loading of shared libraries and the resolving of symbols in them. There are no
instances of this class, it simply serves as a namespace for its static member
functions.

The terms {\it DLL} and {\it shared library/object} will both be used in the
documentation to refer to the same thing: a {\tt .dll} file under Windows or 
{\tt .so} or {\tt .sl} one under Unix.

Example of using this class to dynamically load {\tt strlen()} function:

\begin{verbatim}
#if defined(__WXMSW__)
    static const wxChar *LIB_NAME = _T("kernel32");
    static const wxChar *FUNC_NAME = _T("lstrlenA");
#elif defined(__UNIX__)
    static const wxChar *LIB_NAME = _T("/lib/libc-2.0.7.so");
    static const wxChar *FUNC_NAME = _T("strlen");
#endif

    wxDllType dllHandle = wxDllLoader::LoadLibrary(LIB_NAME);
    if ( !dllHandle )
    {
        ... error ...
    }
    else
    {
        typedef int (*strlenType)(char *);
        strlenType pfnStrlen = (strlenType)wxDllLoader::GetSymbol(dllHandle, FUNC_NAME);
        if ( !pfnStrlen )
        {
            ... error ...
        }
        else
        {
            if ( pfnStrlen("foo") != 3 )
            {
                ... error ...
            }
            else
            {
                ... ok! ...
            }
        }

        wxDllLoader::UnloadLibrary(dllHandle);
    }
\end{verbatim}

\wxheading{Derived from}

No base class

\wxheading{Include files}

<wx/dynlib.h>

\wxheading{Data structures}

This header defines a platfrom-dependent {\tt wxDllType} typedef which stores
a handle to a loaded DLLs on the given platform.

\latexignore{\rtfignore{\wxheading{Members}}}

\membersection{wxDllLoader::LoadLibrary}\label{wxdllloaderloadlibrary}

\func{wxDllType}{LoadLibrary}{\param{const wxString \& }{libname}, \param{bool* }{success = NULL}}

This function loads a shared library into memory, with {\it libname} being the
name of the library: it may be either the full name including path and
(platform-dependent) extenesion, just the basename (no path and no extension)
or a basename with extentsion. In the last two cases, the library will be
searched in all standard locations.

Returns a handle to the loaded DLL. Use {\it success} parameter to test if it
is valid. If the handle is valid, the library must be unloaded later with 
\helpref{UnloadLibrary}{wxdllloaderunloadlibrary}.

\wxheading{Parameters}

\docparam{libname}{Name of the shared object to load.}

\docparam{success}{May point to a bool variable which will be set to TRUE or
FALSE; may also be {\tt NULL}.}

\membersection{wxDllLoader::UnloadLibrary}\label{wxdllloaderunloadlibrary}

\func{void}{UnloadLibrary}{\param{wxDllType }{dllhandle}}

This function unloads the shared library. The handle {\it dllhandle} must have
been returned by \helpref{LoadLibrary}{wxdllloaderloadlibrary} previously.

\membersection{wxDllLoader::GetProgramHandle}\label{wxdllloadergetprogramhandle}

\func{wxDllType}{GetProgramHandle}{\void}

This function returns a valid handle for the main program itself. Notice that
the {\tt NULL} return value is valid for some systems (i.e. doesn't mean that
the function failed).

{\bf NB:} This function is Unix specific. It will always fail under Windows
or OS/2.

\membersection{wxDllLoader::GetSymbol}\label{wxdllloadergetsymbol}

\func{void *}{GetSymbol}{\param{wxDllType }{dllHandle}, \param{const wxString\& }{name}}

This function resolves a symbol in a loaded DLL, such as a variable or
function name.

Returned value will be {\tt NULL} if the symbol was not found in the DLL or if
an error occured.

\wxheading{Parameters}

\docparam{dllHandle}{Valid handle previously returned by 
\helpref{LoadLibrary}{wxdllloaderloadlibrary}}

\docparam{name}{Name of the symbol.}

