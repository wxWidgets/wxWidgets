\section{\class{wxContextHelp}}\label{wxcontexthelp}

This class changes the cursor to a query and puts the application into a 'context-sensitive help mode'.
When the user left-clicks on a window within the specified window, a wxEVT\_HELP event is
sent to that control, and the application may respond to it by popping up some help.

For example:

\begin{verbatim}
  wxContextHelp contextHelp(myWindow);
\end{verbatim}

\wxheading{Derived from}

\helpref{wxObject}{wxobject}

\wxheading{Include files}

<wx/cshelp.h>

\wxheading{See also}

\helpref{wxHelpEvent}{wxhelpevent},\rtfsp
\helpref{wxHelpController}{wxhelpcontroller}

\latexignore{\rtfignore{\wxheading{Members}}}

\membersection{wxContextHelp::wxContextHelp}

\func{}{wxContextHelp}{\param{wxWindow*}{ window = NULL}, \param{bool}{ doNow = TRUE}}

Constructs a context help object, calling \helpref{BeginContextHelp}{wxcontexthelpbegincontexthelp} if\rtfsp
{\it doNow} is TRUE (the default).

If {\it window} is NULL, the top window is used.

\membersection{wxContextHelp::\destruct{wxContextHelp}}

\func{}{\destruct{wxContextHelp}}{\void}

Destroys the context help object.

\membersection{wxContextHelp::BeginContextHelp}\label{wxcontexthelpbegincontexthelp}

\func{bool}{BeginContextHelp}{\param{wxWindow*}{ window = NULL}}

Puts the application into context-sensitive help mode. {\it window} is the window
which will be used to catch events; if NULL, the top window will be used.

Returns TRUE if the application was successfully put into context-sensitive help mode.
This function only returns when the event loop has finished.

\membersection{wxContextHelp::EndContextHelp}\label{wxcontexthelpendcontexthelp}

\func{bool}{EndContextHelp}{\void}

Ends context-sensitive help mode. Not normally called by the application.

