%%%%%%%%%%%%%%%%%%%%%%%%%%%%%%%%%%%%%%%%%%%%%%%%%%%%%%%%%%%%%%%%%%%%%%%%%%%%%%%
%% Name:        stackframe.tex
%% Purpose:     wxStackFrame documentation
%% Author:      Vadim Zeitlin
%% Created:     2005-01-19
%% RCS-ID:      $Id$
%% Copyright:   (c) 2005 Vadim Zeitlin
%% License:     wxWindows license
%%%%%%%%%%%%%%%%%%%%%%%%%%%%%%%%%%%%%%%%%%%%%%%%%%%%%%%%%%%%%%%%%%%%%%%%%%%%%%%

\section{\class{wxStackFrame}}\label{wxstackframe}


wxStackFrame represents a single stack frame, or a single function in the call
stack, and is used exclusively together with 
\helpref{wxStackWalker}{wxstackwalker}, see there for a more detailed
discussion.


\wxheading{Derived from}

No base class

\wxheading{Include files}

<wx/stackwalk.h>

Only available if \texttt{wxUSE\_STACKWALKER} is $1$, currently only
implemented for Win32 and Unix versions using recent version of GNU libc.

\wxheading{See also}

\helpref{wxStackWalker}{wxstackwalker}


\latexignore{\rtfignore{\wxheading{Members}}}


\membersection{wxStackFrame::GetAddress}\label{wxstackframegetaddress}

\constfunc{void*}{GetAddress}{\void}

Return the address of this frame.


\membersection{wxStackFrame::GetFileName}\label{wxstackframegetfilename}

\constfunc{wxString}{GetFileName}{\void}

Return the name of the file containing this frame, empty if
unavailable (typically because debug info is missing).

Use \helpref{HasSourceLocation}{wxstackframehassourcelocation} to check whether
the file name is available.


\membersection{wxStackFrame::GetLevel}\label{wxstackframegetlevel}

\constfunc{size\_t}{GetLevel}{\void}

Get the level of this frame (deepest/innermost one is $0$).


\membersection{wxStackFrame::GetLine}\label{wxstackframegetline}

\constfunc{size\_t}{GetLine}{\void}

Return the line number of this frame, $0$ if unavailable.

\wxheading{See also}

\helpref{GetFileName}{wxstackframegetfilename}


\membersection{wxStackFrame::GetModule}\label{wxstackframegetmodule}

\constfunc{wxString}{GetModule}{\void}

Get the module this function belongs to (empty if not available).


\membersection{wxStackFrame::GetName}\label{wxstackframegetname}

\constfunc{wxString}{GetName}{\void}

Return the unmangled (if possible) name of the function containing this
frame.


\membersection{wxStackFrame::GetOffset}\label{wxstackframegetoffset}

\constfunc{size\_t}{GetOffset}{\void}

Return the return address of this frame.


\membersection{wxStackFrame::GetParam}\label{wxstackframegetparam}

\constfunc{bool}{GetParam}{\param{size\_t }{n}, \param{wxString * }{type}, \param{wxString * }{name}, \param{wxString * }{value}}

Get the name, type and value (in text form) of the given parameter.
Any pointer may be \texttt{NULL} if you're not interested in the corresponding
value.

Return \true if at least some values could be retrieved.

This function currently is only implemented under Win32 and requires a PDB
file.


\membersection{wxStackFrame::GetParamCount}\label{wxstackframegetparamcount}

\constfunc{size\_t}{GetParamCount}{\void}

Return the number of parameters of this function (may return $0$ if we
can't retrieve the parameters info even although the function does have
parameters).


\membersection{wxStackFrame::HasSourceLocation}\label{wxstackframehassourcelocation}

\constfunc{bool}{HasSourceLocation}{\void}

Return \true if we have the file name and line number for this frame.


