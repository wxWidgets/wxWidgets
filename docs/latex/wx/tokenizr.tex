\section{\class{wxStringTokenizer}}\label{wxstringtokenizer}

wxStringTokenizer helps you to break a string up into a number of tokens.

To use this class, you should create a wxStringTokenizer object, give it the
string to tokenize and also the delimiters which separate tokens in the string
(by default, white space characters will be used).

Then \helpref{GetNextToken}{wxstringtokenizergetnexttoken} may be called
repeatedly until it \helpref{HasMoreTokens}{wxstringtokenizerhasmoretokens} 
returns FALSE.

For example:

\begin{verbatim}

wxStringTokenizer tkz("first:second:third::fivth", ":");
while ( tkz.HasMoreTokens() )
{
    wxString token = tkz.GetNextToken();

    // process token here
}
\end{verbatim}

Another feature of this class is that it may return the delimiter which
was found after the token with it. In a simple case like above, you are not
interested in this because the delimiter is always {\tt ':'}, but if the
delimiters string has several characters, you might need to know which of them
follows the current token. In this case, pass {\tt TRUE} to wxStringTokenizer
constructor or \helpref{SetString}{wxstringtokenizersetstring} method and
the delimiter will be appended to each returned token (except for the last
one).

\wxheading{Derived from}

\helpref{wxObject}{wxobject}

\wxheading{Include files}

<wx/tokenzr.h>

\latexignore{\rtfignore{\wxheading{Members}}}

\membersection{wxStringTokenizer::wxStringTokenizer}\label{wxstringtokenizerwxstringtokenizer}

\func{}{wxStringTokenizer}{\void}

Default constructor.

\func{}{wxStringTokenizer}{\param{const wxString\& }{to\_tokenize}, \param{const wxString\& }{delims = " $\backslash$t$\backslash$r$\backslash$n"}, \param{bool }{ret\_delim = FALSE}}

Constructor. Pass the string to tokenize, a string containing delimiters,
a flag specifying whether to return delimiters with tokens.

\membersection{wxStringTokenizer::\destruct{wxStringTokenizer}}\label{wxstringtokenizerdtor}

\func{}{\destruct{wxStringTokenizer}}{\void}

Destructor.

\membersection{wxStringTokenizer::CountTokens}\label{wxstringtokenizercounttokens}

\constfunc{int}{CountTokens}{\void}

Returns the number of tokens in the input string.

\membersection{wxStringTokenizer::HasMoreTokens}\label{wxstringtokenizerhasmoretokens}

\constfunc{bool}{HasMoreTokens}{\void}

Returns TRUE if the tokenizer has further tokens.

\membersection{wxStringTokenizer::GetNextToken}\label{wxstringtokenizergetnexttoken}

\constfunc{wxString}{GetNextToken}{\void}

Returns the next token or empty string if the end of string was reached.

\membersection{wxStringTokenizer::GetPosition}\label{wxstringtokenizergetposition}

\constfunc{size\_t}{GetPosition}{\void}

Returns the current position (i.e. one index after the last returned
token or 0 if GetNextToken() has never been called) in the original
string.

\membersection{wxStringTokenizer::GetString}\label{wxstringtokenizergetstring}

\constfunc{wxString}{GetString}{\void}

Returns the part of the starting string without all token already extracted.

\membersection{wxStringTokenizer::SetString}\label{wxstringtokenizersetstring}

\func{void}{SetString}{\param{const wxString\& }{to\_tokenize}, \param{const wxString\& }{delims = " $\backslash$t$\backslash$r$\backslash$n"}, \param{bool }{ret\_delim = FALSE}}

Initializes the tokenizer.

Pass the string to tokenize, a string containing delimiters,
a flag specifying whether to return delimiters with tokens.

